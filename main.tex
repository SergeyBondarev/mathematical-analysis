%%%%%%%%%%%%%%%%%%%%%%%%%%%%%%%%%%%%%%%%%%%%%%%%%%%%%%%%%%%%%%%%%%%

\documentclass[12pt, russian]{article}
\usepackage[T2A]{fontenc}
\usepackage[utf8]{inputenc}
\usepackage{babel}
\usepackage{amsmath, amssymb, eucal}
\usepackage{ntheorem}

\usepackage[pdftex,unicode, bookmarks, pagebackref]{hyperref}%colorlinks,


\voffset=-10mm
\oddsidemargin=5mm
\evensidemargin=0mm
\textheight=235mm
\textwidth=170mm
\topmargin=-7.2mm

%%%%%%%%%%%%%%%%%%%%%%%%%%%%%%%%%%%%%%%%%%%%%%%%%%%%%%%%%%%%%%%%%%%

\makeatletter

%%%%%%%%%%%%%%%%%%%%%%%%%%%%%%%%%%%%%%%%%%%%%%%%%%%%%%%%%%%%%%%%%%%

\renewcommand{\section}{\@startsection
{section}%
{1}%
{\parindent}%
{2\baselineskip}%
{\baselineskip}%
{\centering\normalfont\Large\bfseries}}

\renewcommand\subsection{\@startsection {subsection}{2}{\parindent}%
{3.5ex plus 1ex minus .2ex}%
{2.3ex plus.2ex}%
{\normalfont\large\bfseries}}

\renewcommand\subsubsection{\@startsection{subsubsection}{3}{\parindent}%
	{3.25ex\@plus 1ex \@minus .2ex}%
	{1.5ex \@plus .2ex}%
	{\normalfont\large\bfseries}}


%%%%%%%%%%%%%%%%%%%%%%%%%%%%%%%%%%%%%%%%%%%%%%%%%%%%%%%%%%%%%%%%%%%

\makeatother

%%%%%%%%%%%%%%%%%%%%%%%%%%%%%%%%%%%%%%%%%%%%%%%%%%%%%%%%%%%%%%%%%%%

\theoremseparator{.}
\newtheorem{theorem}{\hskip 0.5 cm Теорема}%[section]
\newtheorem{definition}{\hskip 0.5 cm Определение}%[section]
\newtheorem{corollary}{\hskip 0.5 cm Следствие}%[section]
\newtheorem{lemma}{\hskip 0.5 cm Лемма}%[section]
\theorembodyfont{\rmfamily}
\newtheorem{remark}{\hskip 0.5 cm Замечание}%[section]
\newtheorem{example}{\hskip 0.5 cm Пример}%[section]
\newtheorem{exercise}{\hskip 0.5 cm Упражнение}%[section]

%%%%%%%%%%%%%%%%%%%%%%%%%%%%%%%%%%%%%%%%%%%%%%%%%%%%%%%%%%%%%%%%%%%

\let\le=\leqslant
\let\ge=\geqslant

\newcommand{\eps}{\varepsilon}

\newcommand{\bbR}{\mathbb{R}}
\newcommand{\bbN}{\mathbb{N}}
\newcommand{\bbZ}{\mathbb{Z}}
\newcommand{\bbQ}{\mathbb{Q}}
\newcommand{\bbC}{\mathbb{C}}


\newcommand{\abs}[1]{\left| #1 \right|}
\newcommand{\brac}[1]{\left(#1\right)}
\newcommand{\fbrac}[1]{\left\{#1\right\}}
\newcommand{\sbrac}[1]{\left[#1\right]}
\newcommand{\abrac}[1]{\langle#1\rangle}

%%%%%%%%%%%%%%%%%%%%%%%%%%%%%%%%%%%%%%%%%%%%%%%%%%%%%%%%%%%%%%%%%%%

\begin{document}
\sloppy
\large

%%%%%%%%%%%%%%%%%%%%%%%%%%%%%%%%%%%%%%%%%%%%%%%%%%%%%%%%%%%%%%%%%%%

\newpage
\tableofcontents
\newpage

%%%%%%%%%%%%%%%%%%%%%%%%%%%%%%%%%%%%%%%%%%%%%%%%%%%%%%%%%%%%%%%%%%%
%%%%%%%%%%%%%%%%%%%%%%%%%%%%%%%%%%%%%%%%%%%%%%%%%%%%%%%%%%%%%%%%%%%
%%%%%%%%%%%%%%%%%%%%%%%%%%%%%%%%%%%%%%%%%%%%%%%%%%%%%%%%%%%%%%%%%%%

\section*{Введение}\addcontentsline{toc}{section}{Введение}
\hskip 1cm

%%%%%%%%%%%%%%%%%%%%%%%%%%%%%%%%%%%%%%%%%%%%%%%%%%%%%%%%%%%%%%%%%%%



%%%%%%%%%%%%%%%%%%%%%%%%%%%%%%%%%%%%%%%%%%%%%%%%%%%%%%%%%%%%%%%%%%%
%%%%%%%%%%%%%%%%%%%%%%%%%%%%%%%%%%%%%%%%%%%%%%%%%%%%%%%%%%%%%%%%%%%
%%%%%%%%%%%%%%%%%%%%%%%%%%%%%%%%%%%%%%%%%%%%%%%%%%%%%%%%%%%%%%%%%%%

\newpage
\section*{Лекция 1. Предварительные сведения.}

\subsection{Понятие множества. Операции над множествами.}

Понятие <<множество>> входит в базовый словарь современной математики. Однако дать строгое определение этому понятию вовсе не так просто. В то же время каждый из нас интуитивно понимает, что \textbf{множество} -- это некоторая \textit{совокупность} каких-то объектов. Важно при этом также понимать, что данное выше <<определение>> вовсе таковым не является, поскольку значение слова <<совокупность>> (ровно как и <<множество>>) так и не было определено.

На данном этапе нам будет удобно оставить понятие множества неопределяемым (или, как говорят, остаться в рамках наивной теории множеств). Позднее мы строго формализуем это понятие, рассмотрев аксиоматический подход к теории множеств.

Слова <<класс>>, <<набор>>, <<совокупность>>, <<семейство>> будут использоваться как синонимы к слову <<множество>>.

%%%%%%%%%%%%%%%%%%%%%%%%%%%%%%%%%%%%%%%%%%%%%%%%%%%%%%%%%%%%%%%%%%%


%%%%%%%%%%%%%%%%%%%%%%%%%%%%%%%%%%%%%%%%%%%%%%%%%%%%%%%%%%%%%%%%%%%
%%%%%%%%%%%%%%%%%%%%%%%%%%%%%%%%%%%%%%%%%%%%%%%%%%%%%%%%%%%%%%%%%%%
%%%%%%%%%%%%%%%%%%%%%%%%%%%%%%%%%%%%%%%%%%%%%%%%%%%%%%%%%%%%%%%%%%%

\newpage
\begin{thebibliography}{99}


\end{thebibliography}

%%%%%%%%%%%%%%%%%%%%%%%%%%%%%%%%%%%%%%%%%%%%%%%%%%%%%%%%%%%%%%%%%%%

\end{document}
