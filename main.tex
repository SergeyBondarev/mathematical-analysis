%%%%%%%%%%%%%%%%%%%%%%%%%%%%%%%%%%%%%%%%%%%%%%%%%%%%%%%%%%%%%%%%%%%

\documentclass[12pt, russian]{article}
\usepackage[T2A]{fontenc}
\usepackage[utf8]{inputenc}
\usepackage{babel}
\usepackage{amsmath, amssymb, eucal}
\usepackage{ntheorem}

\usepackage[pdftex,unicode, bookmarks, pagebackref]{hyperref}%colorlinks,

\renewcommand{\thesection}{Лекция \arabic{section}.}
\renewcommand{\thesubsection}{\arabic{section}.\arabic{subsection}}


\voffset=-10mm
\oddsidemargin=5mm
\evensidemargin=0mm
\textheight=235mm
\textwidth=170mm
\topmargin=-7.2mm

%%%%%%%%%%%%%%%%%%%%%%%%%%%%%%%%%%%%%%%%%%%%%%%%%%%%%%%%%%%%%%%%%%%

\makeatletter

%%%%%%%%%%%%%%%%%%%%%%%%%%%%%%%%%%%%%%%%%%%%%%%%%%%%%%%%%%%%%%%%%%%

\renewcommand{\section}{\@startsection
{section}%
{1}%
{\parindent}%
{2\baselineskip}%
{\baselineskip}%
{\centering\normalfont\Large\bfseries}}

\renewcommand\subsection{\@startsection {subsection}{2}{\parindent}%
{3.5ex plus 1ex minus .2ex}%
{2.3ex plus.2ex}%
{\normalfont\large\bfseries}}

\renewcommand\subsubsection{\@startsection{subsubsection}{3}{\parindent}%
	{3.25ex\@plus 1ex \@minus .2ex}%
	{1.5ex \@plus .2ex}%
	{\normalfont\large\bfseries}}


%%%%%%%%%%%%%%%%%%%%%%%%%%%%%%%%%%%%%%%%%%%%%%%%%%%%%%%%%%%%%%%%%%%

\makeatother

%%%%%%%%%%%%%%%%%%%%%%%%%%%%%%%%%%%%%%%%%%%%%%%%%%%%%%%%%%%%%%%%%%%

\theoremseparator{.}
\newtheorem{theorem}{\hskip 0.5 cm Теорема}%[section]
\newtheorem{definition}{\hskip 0.5 cm Определение}%[section]
\newtheorem{corollary}{\hskip 0.5 cm Следствие}%[section]
\newtheorem{lemma}{\hskip 0.5 cm Лемма}%[section]
\theorembodyfont{\rmfamily}
\newtheorem{remark}{\hskip 0.5 cm Замечание}%[section]
\newtheorem{example}{\hskip 0.5 cm Пример}%[section]
\newtheorem{exercise}{\hskip 0.5 cm Упражнение}%[section]

%%%%%%%%%%%%%%%%%%%%%%%%%%%%%%%%%%%%%%%%%%%%%%%%%%%%%%%%%%%%%%%%%%%

\let\le=\leqslant
\let\ge=\geqslant

\newcommand{\eps}{\varepsilon}

\newcommand{\bbR}{\mathbb{R}}
\newcommand{\bbN}{\mathbb{N}}
\newcommand{\bbZ}{\mathbb{Z}}
\newcommand{\bbQ}{\mathbb{Q}}
\newcommand{\bbC}{\mathbb{C}}


\newcommand{\abs}[1]{\left| #1 \right|}
\newcommand{\brac}[1]{\left(#1\right)}
\newcommand{\fbrac}[1]{\left\{#1\right\}}
\newcommand{\sbrac}[1]{\left[#1\right]}
\newcommand{\abrac}[1]{\langle#1\rangle}

%%%%%%%%%%%%%%%%%%%%%%%%%%%%%%%%%%%%%%%%%%%%%%%%%%%%%%%%%%%%%%%%%%%

\begin{document}
\sloppy
\large

%%%%%%%%%%%%%%%%%%%%%%%%%%%%%%%%%%%%%%%%%%%%%%%%%%%%%%%%%%%%%%%%%%%

\newpage
\tableofcontents
\newpage

%%%%%%%%%%%%%%%%%%%%%%%%%%%%%%%%%%%%%%%%%%%%%%%%%%%%%%%%%%%%%%%%%%%
%%%%%%%%%%%%%%%%%%%%%%%%%%%%%%%%%%%%%%%%%%%%%%%%%%%%%%%%%%%%%%%%%%%
%%%%%%%%%%%%%%%%%%%%%%%%%%%%%%%%%%%%%%%%%%%%%%%%%%%%%%%%%%%%%%%%%%%

\section*{Введение}\addcontentsline{toc}{section}{Введение}
\hskip 1cm

%%%%%%%%%%%%%%%%%%%%%%%%%%%%%%%%%%%%%%%%%%%%%%%%%%%%%%%%%%%%%%%%%%%



%%%%%%%%%%%%%%%%%%%%%%%%%%%%%%%%%%%%%%%%%%%%%%%%%%%%%%%%%%%%%%%%%%%
%%%%%%%%%%%%%%%%%%%%%%%%%%%%%%%%%%%%%%%%%%%%%%%%%%%%%%%%%%%%%%%%%%%
%%%%%%%%%%%%%%%%%%%%%%%%%%%%%%%%%%%%%%%%%%%%%%%%%%%%%%%%%%%%%%%%%%%

\newpage
\section{Предварительные сведения.}

\subsection{Введение в математическую логику.}

\subsubsection{Алгебра высказываний.}

Понятие \textbf{высказывания} является первоначальным и не определяется. Можно лишь пояснить, что \textbf{высказыванием} мы будем называть любое повествовательное предложение, про которое определенно можно сказать, истинно оно или ложно. Высказывание может быть записано как на естественном языке, так и с помощью <<специального>> языка (например, математического).

\begin{example}
	<<Простых чисел бесконечно много>> --- верное высказывание. <<$2+2=5$>> --- неверное высказывание. Фраза <<математическая логика --- это скучно>> высказыванием не является. <<$x^2 + 5x - 6 \ge 0$>> также не является высказыванием.
\end{example}

\textbf{Значением истинности} или \textbf{индикаторным значением} высказывания $P$ будем называть число, равное $1$ в случае истинности $P$ и равное $0$, когда $P$ ложно.

В естественном языке можно образовать новые высказывания при помощи связок <<не>>, <<и>>, <<или>>, <<если ..., то ...>>. При помощи значений истинности можно корректно определить высказывания, соответствующие этим связкам.

\textbf{Отрицание} высказывания $P$ --- высказывание, истинное если  и только если $P$ ложно. Отрицание высказывания $P$ обозначают $\neg P$ или $\overline{P}$. При этом $\overline{P}$ читается как <<не $P$>>.

\textbf{Конъюнкция} двух высказываний $P$ и $Q$ --- высказывание, обозначаемое через $P \wedge Q$, истинное тогда и только тогда, когда истинны оба высказывания $P$ и $Q$. $P \wedge Q$ читается как <<$P$ и $Q$>>.

\textbf{Дизъюнкция} двух высказываний $P$ и $Q$ --- высказывание, обозначаемое через $P \vee Q$, истинное тогда и только тогда, когда истинно хотя бы одно из высказываний $P$ и $Q$. $P \vee Q$ читается как <<$P$ или $Q$>>. 

\textbf{Импликацией} высказываний $P$ и $Q$ будем называть высказывание ложное только тогда, когда $P$ истинно, а $Q$ ложно. Импликацию обозначают как $P \implies Q$. Читается это как <<из $P$ следует $Q$>> или <<$P$ влечет $Q$>>.

Как уже было сказано, можно определить данные выше понятие с помощью значений истинности, записав следующую таблицу, которую обычно называют \textbf{таблицей истинности}.

\begin{center}
\begin{tabular}{ |c|c|c|c|c|c| } 
	\hline
		$P$ & $Q$ & $\overline{P}$  & $P\wedge Q$ & $P \vee Q$ & $P \implies Q$\\ \hline
		$0$ & $0$ & $1$ & $0$ & $0$ & $1$ \\ \hline
		$0$ & $1$ & $1$ & $0$ & $1$ & $1$ \\ \hline
		$1$ & $0$ & $0$ & $0$ & $1$ & $0$ \\ \hline
		$1$ & $1$ & $0$ & $1$ & $1$ & $1$ \\
	\hline
\end{tabular}
\end{center}


\subsection{Понятие множества. Операции над множествами.}

Понятие <<множество>> входит в базовый словарь современной математики. Однако дать строгое определение этому понятию вовсе не так просто. В то же время каждый из нас интуитивно понимает, что \textbf{множество} -- это некоторая \textit{совокупность} каких-то объектов. Важно при этом также понимать, что данное выше <<определение>> вовсе таковым не является, поскольку значение слова <<совокупность>> (ровно как и <<множество>>) так и не было определено.

На данном этапе нам будет удобно оставить понятие множества неопределяемым (или, как говорят, остаться в рамках наивной теории множеств). Позднее мы строго формализуем это понятие, рассмотрев аксиоматический подход к теории множеств.

Слова <<класс>>, <<набор>>, <<совокупность>>, <<семейство>> будут использоваться как синонимы к слову <<множество>>.

\begin{example}
	Примерами множеств могут служить множество студентов в аудитории,  множество букв русского алфавита, множество страниц в книге. Как правило, множества мы будем обозначать заглавными латинскими буквами.
\end{example}

\textit{Элементом} множества будем называть объект, принадлежащий этому множеству. Запись $x \in X$ будет означать, что $x$ является элементом множества $X$. Если $x$ не принадлежит $X$, то будем записывать этот факт как $x \notin X$.

Множество, не содержащее элементов, называется \textit{пустым} и обозначается $\varnothing$.

Рассмотрим способы задания множеств.

\begin{enumerate}
	\item \textit{Перечисление}. Чтобы задать множество, нужно попросту указать все элементы, которые ему принадлежат. Например, можно задать множество букв английского алфавита $M$ следующим образом
	$$
	M = \{ a, b, c, d, e, \ldots, x, y, z \}.
	$$

	\item \textit{Характеристический предикат}. Чтобы задать множество, нужно указать предикат $P(x)$, истинный тогда и только тогда, когда $x$ принадлежит множеству.
	$$
	M = \{ x \, | \,  P(x)\} = \{ x: P(x)\}.
	$$

	\item \textit{Индексация}. Это способ используется, если нужно проиндексировать (<<занумеровать>>) элементы одного множества с помощью другого множества --- множества индексов $I$. 
	$$
	M = \{ x_i \}_{i \in I}.
	$$
	Например, если $I=\{2,3,5\}$, то
	$$
	M = \{ x_i \}_{i \in I} = \{ x_2, x_3, x_5 \}.
	$$
\end{enumerate}




\subsubsection{Операции над множествами.}

Будем говорить, что множество $X$ \textbf{содержится} в множестве $Y$ (или что $Y$ содержит $X$), если для любого $x \in X$ выполнено $x \in Y$. Записывать это будем следующим образом: $X \subset Y$. Альтернативный способ прочтения этой записи --- <<$X$ является \textbf{подмножеством} $Y$>>.

Заметим, что пустое множество $\varnothing$ является подмножеством любого множества. Также отметим, что множество всегда является своим подмножеством.

Множества $X$ и $Y$ будем называть равными при выполнении обоих условий
$$
X \subset Y, \; Y \subset X.
$$

\begin{definition}
	\textbf{Пересечением} двух множеств $X$ и $Y$ назовем множество $X \cap Y$, состоящее из элементов, принадлежащих одновременно множествам $X$ и $Y$.
\end{definition}
\begin{definition}
	\textbf{Объединением} двух множеств $X$ и $Y$ назовем множество $X \cup Y$, состоящее из элементов, принадлежащих хотя бы одному из множеств $X$ и $Y$.
\end{definition}
Более формально данные выше определения можно записать как
$$
X \cap Y = \{ x:  (x \in X) \wedge (x \in Y) \},
$$
$$
X \cup Y = \{ x:  (x \in X) \vee (x \in Y) \},
$$
откуда четко видно соответствие между логическими и теоретико-множественными операциями.

\subsection{Понятие функции.}

Пусть $X$, $Y$ --- какие-то множества. Попытаемся определить понятие функции, определенной на $X$ со значениями во множестве $Y$. Сначала рассмотрим обычное <<школьное>> определение.

\begin{definition}[нестрогое]
	Функцией $f$, определенной на $X$ со значениями в $Y$, называется некое \textit{правило} (закон), по которому каждому элементу $x \in X$ сопоставляется некий элемент $y \in Y$. При этом $X$ называется областью определения функции $f$, а $Y$ --- областью значений функции $f$.
\end{definition}

%%%%%%%%%%%%%%%%%%%%%%%%%%%%%%%%%%%%%%%%%%%%%%%%%%%%%%%%%%%%%%%%%%%


%%%%%%%%%%%%%%%%%%%%%%%%%%%%%%%%%%%%%%%%%%%%%%%%%%%%%%%%%%%%%%%%%%%
%%%%%%%%%%%%%%%%%%%%%%%%%%%%%%%%%%%%%%%%%%%%%%%%%%%%%%%%%%%%%%%%%%%
%%%%%%%%%%%%%%%%%%%%%%%%%%%%%%%%%%%%%%%%%%%%%%%%%%%%%%%%%%%%%%%%%%%

\newpage
\begin{thebibliography}{99}


\end{thebibliography}

%%%%%%%%%%%%%%%%%%%%%%%%%%%%%%%%%%%%%%%%%%%%%%%%%%%%%%%%%%%%%%%%%%%

\end{document}
